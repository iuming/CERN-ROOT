\documentclass{article}

\usepackage{ctex}
\usepackage{graphicx}
% \usepackage{minted}   % LaTeX内置代码包
\usepackage{markdown}
\usepackage{listings}

\lstset{
%  columns=fixed,       
%  numbers=left,                                        % 在左侧显示行号
%  numberstyle=\tiny\color{gray},                       % 设定行号格式
%  frame=none,                                          % 不显示背景边框
%  backgroundcolor=\color[RGB]{245,245,244},            % 设定背景颜色
%  keywordstyle=\color[RGB]{40,40,255},                 % 设定关键字颜色
%  numberstyle=\footnotesize\color{darkgray},           
%  commentstyle=\it\color[RGB]{0,96,96},                % 设置代码注释的格式
%  stringstyle=\rmfamily\slshape\color[RGB]{128,0,0},   % 设置字符串格式
 showstringspaces=false,                              % 不显示字符串中的空格
 language=c++,                                        % 设置语言
 breaklines,                                          %自动换行
}

\usepackage[left=2.5cm, right=2.5cm, top=2.8cm, bottom=2.8cm]{geometry}
\usepackage{hyperref}

\title{侯哥自制教程视频NOTES}
\author{刘铭}
\date{\today}

\begin{document}
\maketitle
\tableofcontents
% \begin{markdown}
%     这里面是markdown语法!
% \end{markdown}

\section{ROOT Beginner}
\subsection{使用ROOT作为计算器}
启动ROOT:\texttt{ root -l } ( \texttt{-l}选项指不显示root版本信息)

\paragraph{示例:}
\begin{itemize}
    \item \texttt{sqet(2)}      计算根号2
    \item \texttt{pow(8.100)}   计算8的100次方
\end{itemize}

\subsection{Function}
\textbf{注意:}自建文件夹的字符尽量采用英文字符且不带有特殊字符


Linux中创建文件夹命令:\texttt{mkdir <FloderName>}

创建ROOT文件,一般采用\texttt{.C}结尾的文件名!

执行ROOT文件,采用指令
\begin{enumerate}
    \item \texttt{root -l <FileName>}
    \item \begin{itemize}
              \item 打开root \texttt{root -l}
              \item 执行文件 \texttt{.x <FileName>}
          \end{itemize}
    \item 当一个文件中有不止一个函数时,采用:
          \begin{itemize}
              \item 打开root \texttt{root -l}
              \item 加载文件 \texttt{.L <FileName>}
              \item 执行函数 \texttt{<FunctionName>}
          \end{itemize}
\end{enumerate}

Linux所有程序里的结构为:\texttt{<Program> <-option> <FileName>}

\subsubsection{TF1 TF2}
\begin{lstlisting}
    void f1(){
        TF1* f1 = new TF1("f1", "sin(x)/x", 0, 10);
        f1->Draw();
    }
\end{lstlisting}

以上代码中\texttt{TF1}类的构造函数输入参数分别为“名字",“内容”,“下限”,“上限”。

绘制执行流程:
\begin{enumerate}
    \item 启动root \texttt{root -l}
    \item 加载文件 \texttt{.L f1.C}
    \item 执行函数 \texttt{f1()}
\end{enumerate}

\texttt{TF1}为root中内置类,代表一维函数。同理,\texttt{TF2}为二维函数,例如:
\begin{lstlisting}
    void f2(){
        TF2* f2 = new TF2("f2", "sin(x)/x+y*y", 0, 10, 0, 10);
        f2->Draw("surf");
    }
\end{lstlisting}

若在\texttt{Draw()}函数中不添加\texttt{surf}参数,则绘制出来为“等高线”!\texttt{surf}为\texttt{Draw()}函数中的参数,可以从ROOT官网中查询其他参数!

添加参数的函数:
\begin{lstlisting}
    void ff(){
        TF1* f3 = new TF1("f3", "[0]*sin(x)/x+[1]", 0, 10);
        f3->SetParameter(0,1);
        f3->SetParameter(1,1);
        f3->Draw();
    }
\end{lstlisting}

设置函数参数的方式为\texttt{SetParameter(<ParaNum>,<value>)}
\newline
ROOT中已定义的函数:
\begin{enumerate}
    \item \texttt{gaus(<mean>, <sigma>)}
\end{enumerate}

Function在拟合中使用较多,其他地方使用较少!

\subsubsection{Graph}
\begin{enumerate}
    \item 直接定义各个points:
          \begin{lstlisting}
    void Graph(){
        TGraph* g1 = new TGraph();
        g1->SetPoint(0, 0, 1);
        g1->SetPoint(1, 1, 3);
        g1->SetPoint(2, 2, 6);
        g1->SetPoint(3, 3, 1);
        g1->Draw();
    }
\end{lstlisting}

          Graph的本质是由有限个点构造出来的图形,若不设置第$ 0 $个点,则第$ 0 $个点坐标为$ \left( 0, 0 \right) $。

          Graph中\texttt{Draw()}函数中采用参数\texttt{A*L},则绘制出点显示为$ * $的折线图;若参数为\texttt{A*C},则绘制出点为$ * $的曲线图。
    \item Graph读取文件:
          \begin{lstlisting}
    void Graph2(){
        TGraph* g2 = new TGraph("data.txt", "%lg %lg");
        g2->Draw("A*L");
    }
\end{lstlisting}
          其中,\texttt{data.txt}文件中数据存储的格式为\texttt{x   y};
    \item TGraph通过数组绘制图形:
          \begin{lstlisting}
    void Graph3(){
        const int N = 5;
        double x[N] = {1, 2, 3, 4, 5};
        double y[N] = {7, 2, 4, 4, 5};
        TGraph* g3 = new TGraph(N, x, y);
        g3->Draw("A*L");
    }
\end{lstlisting}
\end{enumerate}

为TGraph设置标题函数:\texttt{SetTitle("<Title>")}

\subsubsection{TCanvas}
画布的使用:
\begin{enumerate}
    \item 不需要新建,画图时ROOT会自动生成(如上面所有示例);
    \item 使用TCanvas类新建,\texttt{TCanvas* c1 = new TCanvas();}
    \item 创建几个不同的画布:
          \begin{lstlisting}
        void Canvas(){
            const int N = 5;
            double x[N] = {1, 2, 3, 4, 5};
            double y[N] = {7, 2, 4, 4, 5};
            TCanvas* c2 = new TCanvas("c2", "c2");
            TCanvas* c3 = new TCanvas("c3", "c3");
            c2->cd();
            TGraph* g4 = new TGraph(N, x, y);
            g4->Draw("A*L");
            c3->cd();
            TGraph* g5 = new TGraph("data.txt", "%lg %lg");
            g5->Draw("A*L");
        }
    \end{lstlisting}
          上面例子中\texttt{TCanvas}的构造函数为\texttt{TCanvas("<Name>", "<Title>")};
    \item Canvas分区域:
          \begin{lstlisting}
        void Canvas2(){
            TCanvas* c4 = new TCanvas("c4", "c4");
            c4->Divide(2, 2);
            TGraph* g6 = new TGraph("data.txt", "%lg %lg");
            c4->cd(1);
            g6->Draw("A*L");
            c4->cd(2);
            g6->Draw("A*C");
            c4->cd(3);
            g6->Draw("A0L");
            c4->cd(4);
            g6->Draw("A0C");
        }
    \end{lstlisting}
\end{enumerate}

画布菜单栏中可直接编辑图形内容,例如编辑形状、更改颜色、PointStyle、坐标轴等等。所有画布菜单栏中完成的事情,使用代码都能完成!

\subsubsection{Histogram}
核物理中最常用的类!

\begin{enumerate}
    \item 简单定义:
          \begin{lstlisting}
    void Hist(){
        int nbins = 100;
        double xmin = 0.;
        double xmax = 100.;
        TH1D* h1 = new TH1D("h1", "title", nbins, xmin, xmax);
        h1->Fill(50);
        h1->Fill(50);
        h1->Fill(20);
        h1->Draw();
    }
\end{lstlisting}
          \texttt{TH1D}类的构造函数的定义为\texttt{TH1D("<Name>", "<Title>", <nbins>, <xmin>, <xmax>);}

          一维直方图类型有\texttt{TH1F}(浮点数)、\texttt{TH1D}(双精度)、\texttt{TH1I}(整形)、\texttt{TH1S}、\texttt{TH1C}(字符串)、\texttt{TH1K}等。
    \item 填充随机数:
          \begin{lstlisting}
    void Hist2(){
        int nbins = 100;
        double xmin = -10.;
        double xman = 10.;
        TH1D* h2 = new TH1D("h2", "title", nbins, xmin, xmax);
        TRandom3* rng = new TRandom3();
        for(int i=0; i< 10000; ++i){
            double r = rng->Gaus(0, 1);
            h2->Fill(r);
        }
        h2->Draw();
    }
\end{lstlisting}
          采用一维直方图可以用来显示时间分布、位置分布等等!
    \item 二维直方图:
          \begin{lstlisting}
    void Hist3(){
        int nbins = 100;
        double xmin = -10.;
        double xman = 10.;
        double ymin = 
        double ymax = 
        TH2D* h3 = new TH2D("h3", "title", nbins, xmin, xmax, nbins, ymin, ymax);
        for(int i=0; i< 1000000; ++i){
            double x = rng->Gaus(0, 1);
            double y = rng->Gaus(0, 2);
            h3->Fill(x, y);
        }
        h3->Draw("surf");
        gStyle->Setpalette(1);
    }
\end{lstlisting}
          二维直方图中\texttt{Draw()}函数中的参数有\texttt{surf,surf1,surf2,surf3,lego,lego2,lego3}等。常用的是\texttt{COLZ},用颜色代表其计数,颜色越亮,代表数据越多!上面示例中\texttt{gStyle->SetPalette(<StyleNum>)}函数含义为颜色尺类型,更改其数字可以更换不同颜色盘!
\end{enumerate}

\subsubsection{Fit}
做模拟时得到数据,例如位置分布,有了位置分布之后需要获得其位置,需要拟合其高斯分布的中心。
\newline
拟合的方式有:
\begin{enumerate}
    \item 画布菜单栏选择Fit;
    \item 代码:
          \begin{lstlisting}
        void fit(){
            int nbins = 100;
            double xmin = -10.;
            double xman = 10.;
            TH1D* h4 = new TH1D("h4", "title", nbins, xmin, xmax);
            TRandom3* rng = new TRandom3();
            for(int i=0; i< 10000; ++i){
                double r = rng->Gaus(0, 1);
                h4->Fill(r);
            }
            h4->Fit("gaus");
        }
    \end{lstlisting}
          拟合函数可以拟合高斯分布、指数分布、朗道分布、多项式分布等!也可以自定义函数进行拟合!
    \item 自定义函数拟合:
          \begin{lstlisting}
        void fit2(){
            int nbins = 100;
            double xmin = -10.;
            double xman = 10.;
            TF1* f4 = new TF1("f4", "[0]*sin(x)/x+[1]", 0, 10);
            TH1D* h5 = new TH1D("h5", "title", nbins, xmin, xmax);
            TRandom3* rng = new TRandom3();
            for(int i=0; i< 10000; ++i){
                double r = rng->Gaus(0, 1);
                h5->Fill(r);
            }
            h5->Fit("f4");
        }
    \end{lstlisting}
    \item 获得拟合参数:
          \begin{lstlisting}
        void fit3(){
            int nbins = 100;
            double xmin = -10.;
            double xman = 10.;
            TF1* f5 = new TF1("f5", "[0]*sin(x)/x+[1]", 0, 10);
            TH1D* h6 = new TH1D("h6", "title", nbins, xmin, xmax);
            TRandom3* rng = new TRandom3();
            for(int i=0; i< 10000; ++i){
                double r = rng->Gaus(0, 1);
                h6->Fill(r);
            }
            h6->Fit("f5");
            cout << "p0=" << f5->GetParameter(0) << "\t p1=" << f5->GetParameter(1) << endl;
        }
    \end{lstlisting}
    \item 分段函数拟合:
          \begin{lstlisting}
        void fun(){
            if(x > 0)
            
            else
        }
    \end{lstlisting}
          通过定义\texttt{fun()}函数来定义分段函数,然后进行拟合!
\end{enumerate}


\section{Geant4 Beginner}
Geant4二次开发目录\texttt{sg410mt},运行目录\texttt{tpt},编译指令\texttt{. build.sh},编译完成后会自动进入\texttt{bin}目录下,执行指令\texttt{./tpt}。

一般执行前先执行\texttt{. clear.sh},将之前编译的cmake文件删除!

使用之前将\texttt{build.sh}文件中Geant4工作目录更改为当前计算机Geant安装目录。

\subsection{探测器设置}
建模源文件:\texttt{sg4Detector.cc}
建模的步骤:
\begin{enumerate}
    \item 定义材料;
          \begin{itemize}
              \item 从Geant4库中找Geant4定义好的材料:
                    \begin{lstlisting}
            G4NistManager* nist = G4NistManager::Instance();
            Air = nist->FindOrBuildMaterial("G4_AIR");
        \end{lstlisting}
                    可以在Geant4官网手册找到Geant4定义的材料!
              \item 自己定义材料:
                    \begin{lstlisting}
                  //Vacuum
                  new G4Material("vacuum", z=1., a=1.01*g/mole, density=universe_mean_density, kStateGas, 2.73*kelvin, 3.e-18*pascal);

                  //Scitillator(BC408)
                  BC408 = new G4Material("BC408", density=1.03*g/mole, nel=2);
                  BC408->AddElement(C, natoms=10);
                  BC408->AddElement(H, natoms=11);
              \end{lstlisting}
                    可定义单质、化合物、混合物等材料,具体定义可参考Geant4官网样例以及文档!
          \end{itemize}
    \item 定义几何体,具体步骤如下:
          \begin{itemize}
              \item 通过\texttt{G4Material::GetMaterial()}获取\texttt{DefineMaterial()}中定义的材料:
                    \begin{lstlisting}
            G4Material* world_mat =  G4Material::GetMaterial("G4_AIR");
        \end{lstlisting}
              \item 定义\texttt{World}:
                    \begin{lstlisting}
             G4Box* soildWorld = new G4Box("world", 0.5*world_x, 0.5*world_y, 0.5*world_y);
             logicWorld = new G4LogicalVolume(soildVolume, world_mat, "world");
             physiWorld = new G4PVPlacement(0, G4ThreeVector(), logicalVolume, "world", 0, false, 0, checkOverlaps);
        \end{lstlisting}
                    建立几何体分为三个步骤,首先建立\texttt{soildVolume},主要定义形状和尺寸;然后建立\texttt{logicalVolume},主要定义材料;最后建立\texttt{PhysicalVolume},主要定义位置和其他几何体相互关系、是否旋转等信息。
                    \newline
                    \texttt{soildVolume}在Geant4中定义过很多形状,可以查询Geant4手册进行学习!
                    \newline
                    \texttt{PhysicalVolume}具体参数可参照Geant4官方手册或者本人之前学习Geant4所做的\href{https://github.com/iuming/Geant4_Learning/blob/16f8670a6f606cbc0eca1225a81669e54e42594a/example/B1/src/B1DetectorConstruction.cc#L90}{记录}。
              \item 定义Envelope,定义方式与World相似,其\texttt{physicalVolume}的母体为World;
              \item 定义实验所需几何体,定义方式与World相似。
              \item 最后选择几何体作为探测器,定义为\texttt{SD}(灵敏探测器)
                    \begin{lstlisting}
                  void sg4Detector::ConstructSDandField(){
                      DefineOneSD(<logicalVolume>);
                  }
              \end{lstlisting}
          \end{itemize}
\end{enumerate}

\subsection{源项设置}
源项设置文件:\texttt{input.txt},该文件为读卡文件!
\newline
设置多线程:\texttt{NumberOfThreads <Number>}

更改源项设置位置:\texttt{primary particle}

粒子种类:\texttt{opticalphoton, neutron, alpha, gamma, ion}

离子参数:\texttt{IonZ(原子序数), IonA(质量数), IonEstar(激发态), IonQ(带电量)}

初射动能:\texttt{Ekprimary <动能个数> <能量>} \\
\hspace*{2.5cm} \texttt{EkWeight <动能个数> <权重>}

粒子入射位置:\texttt{xInitParimary <xPosition>} \\
\hspace*{3.2cm} \texttt{yInitParimary <yPosition>} \\
\hspace*{3.2cm} \texttt{zInitParimary <zPosition>}

源项大小(弥散范围):\texttt{xRangePrimary <xSize>} \\
\hspace*{2.5cm} \texttt{yRangePrimary <ySize>} \\
\hspace*{2.5cm} \texttt{zRangePrimary <zSize>}
\newline
通过设置$ xyz $方面源项的大小,可以设置该放射源为点源、线源、面源或者体源!本质上为源项中心在弥散范围内产生随机数作为粒子发射初始位置!

粒子旋转角(粒子初射方向):\texttt{DirectRotX <Rot>} \\
\hspace*{5.5cm} \texttt{DirectRotY <Rot>} \\
\hspace*{5.5cm} \texttt{DirectRotZ <Rot>}

粒子初射范围(弥散):\texttt{RangeThetaPri <degree>}
\newline
粒子发射范围,若为$ 0 $,则发射角度不变;若为$ 180 $,则代表模拟的粒子为点源,各项同性。

\textbf{额外补充:}粒子源设置在Geant4官方教程中有更多方法,例如视频中中子源设置,采用将中子能谱分割的方式模拟并不精确。在官方教程中,可以将能谱导入Geant4,采用\texttt{GPS(Geant4 Particle Sources)}方式模拟中子源,相比于分割能谱的方式能够更加精确的模拟中子源!

Geant4模拟生成文件名:\texttt{FileNameNt: <FileName>.root}

\subsection{数据处理}
若采用多线程运行Geant4,会产生多个root文件。例如,采用4线程模拟,将会生成\texttt{test\_t1.root},\texttt{test\_t2.root}、\texttt{test\_t3.root}、\texttt{test\_t4.root}四个root文件。Geant4模拟结束后,首先采用\texttt{hadd all.root test\_t*.root}命令来合并所有\texttt{root}文件。

\textbf{额外补充:}Geant4里面对于多线程已开发出自动合并的方法,可见官网教程\href{https://github.com/Geant4/geant4/blob/3dfcdb544e19888c7a5979720ae09596207436f2/examples/extended/analysis/AnaEx01/src/HistoManager.cc#L60}{AnaEx01}。

完成数据合并后,使用ROOT打开root文件:\texttt{root -l all.root};

打开root文件后,可输入\texttt{.ls}查看root文件下有哪些内容;

然后输入\texttt{t->Print()}查看\texttt{tree}下的\texttt{Branch}分支;输入\texttt{t->Scan()}来查看\texttt{Branch}数据。

ROOT文件中\texttt{tree}可以看成是一个表格,其中每个\texttt{Branch}可以看成是一列。\texttt{Branch}中一般存储序号、能量、粒子名字、位置$ (x, y, z) $、时间等。

sg410mt文件中含有Formate.txt,里面包含了\texttt{Branch}信息。

Geant4模拟运行过程可见本人之前学习Geant4的\href{https://blog.csdn.net/sinat_33249803/article/details/116067709?spm=1001.2014.3001.5501}{记录}。


对\texttt{tree}数据处理:
\begin{enumerate}
    \item \texttt{t->MakeSelector("ana")} 会生成\texttt{ana.C}和\texttt{ana.h}两个文件来处理\texttt{tree}的\texttt{Branch}
    \item 编辑\texttt{ana.C}文件中的函数,里面有\texttt{Begin()}、\texttt{SlaveBegin()}、\texttt{Process()}、\texttt{SlaveTerminate()}、\texttt{Terminate()}五个函数。常用的函数为\texttt{Begin()}、\texttt{Process()}、\texttt{Terminate()}三个函数,该类的调用方式为:首先运行\texttt{Begin()}函数,然后重复运行\texttt{Process()}函数,重复次数为Geant4模拟的Event数,结束后运行\texttt{Terminate()}函数。
    \item 一般来说,在\texttt{Begin()}函数中定义直方图,在\texttt{Process()}函数中填充直方图数据,在\texttt{Terminate()}函数中绘制直方图!
    \item 编写完成\texttt{ana}类中函数后,使用ROOT打开\texttt{root}文件,然后输入指令\texttt{tree->Process("ana.C")}执行。
\end{enumerate}

\section{Analysis Example}
\subsection{获取能谱}
\subsection{获取位置分布}
\subsection{符合事件}
\subsection{时间测量}
\subsection{径迹重构}

\end{document}